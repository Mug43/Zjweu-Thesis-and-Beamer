% 引言(独立章节,无编号)

\vspace{\baselineskip}
\begin{center}
\zihao{3}\bfseries\songti 引\quad 言
\end{center}
\vspace{\baselineskip}

\indent 随着科学技术的快速发展,遥感技术已成为地球观测和资源环境监测的重要手段。特别是在水文领域,卫星遥感技术凭借其宏观视野、高时空分辨率和全天候观测能力,为水资源管理、洪涝灾害监测、水环境评估等提供了重要的技术支撑。

近年来,美国、欧洲、中国、日本等国家相继发射了多颗水文遥感卫星,在降水监测、土壤湿度反演、地表水体提取、蒸散发估算等方面取得了显著进展。这些技术突破不仅丰富了水文学的研究手段,也为全球水资源管理提供了新的解决方案。

然而,不同国家的卫星遥感技术在传感器类型、数据质量、应用领域等方面各有特点。系统梳理和比较分析这些技术进展,对于推动我国水文遥感技术发展、提升水资源管理水平具有重要意义。

本研究旨在通过文献综述与案例分析相结合的方法,系统总结不同国家卫星遥感在水文领域的最新进展,分析各国技术特点与应用实践,为我国相关领域的发展提供参考和借鉴。
