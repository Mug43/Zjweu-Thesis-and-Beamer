% 第一章:绪论

\chapter{国外卫星遥感技术进展}

\section{美国水文遥感卫星}
美国在水文遥感领域处于世界领先地位,拥有完善的卫星体系和丰富的应用经验。NASA(美国国家航空航天局)和NOAA(美国国家海洋和大气管理局)是主要的卫星运营机构。

\subsection{TRMM/GPM降水监测卫星}
热带降雨测量任务(TRMM)卫星于1997年发射,配备降水雷达和微波成像仪,首次实现了热带地区降水的三维结构探测。2014年,全球降水测量(GPM)卫星作为TRMM的继承者成功发射,监测范围扩展至全球,时空分辨率显著提升\cite{ref1}。

\subsection{SMAP土壤湿度卫星}
土壤湿度主被动卫星(SMAP)于2015年发射,结合L波段雷达和辐射计,可提供9公里分辨率的全球土壤湿度数据,为农业干旱监测和洪水预报提供了重要支撑。

\section{欧洲Sentinel系列卫星}
欧空局(ESA)的哥白尼计划(Copernicus)是目前全球最大的对地观测计划之一。Sentinel系列卫星为水文应用提供了丰富的免费数据资源。

\subsection{Sentinel-1雷达卫星}
Sentinel-1携带C波段合成孔径雷达(SAR),具有全天候、全天时观测能力,在洪水监测、湿地制图等方面具有独特优势。其6天重访周期为动态水体监测提供了高频数据\cite{ref2}。

\subsection{Sentinel-2光学卫星}
Sentinel-2携带多光谱成像仪,空间分辨率达10-20米,5天重访周期,为地表水体提取、水质监测提供了高质量数据。其红边波段对水生植被识别具有重要价值。

\section{本章小结}
国外卫星遥感技术在传感器性能、数据质量、应用深度等方面处于领先地位。美国强调业务化运行和全球监测能力,欧洲注重数据开放共享和多任务协同,为全球水文研究提供了重要数据支撑。
