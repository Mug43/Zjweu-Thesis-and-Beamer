% 第二章:国内卫星遥感技术进展

\chapter{国内卫星遥感技术进展}

\section{中国气象卫星系统}
中国气象卫星事业起步于20世纪70年代,经过多年发展,已建立起由极轨和静止轨道气象卫星组成的完整业务观测体系。

\subsection{风云系列气象卫星}
风云系列卫星是中国自主研制的气象卫星系统,包括风云一号、二号、三号、四号等多个系列,在天气预报、气候监测、防灾减灾等领域发挥着重要作用。

表\ref{tab:fengyun-satellites}列出了主要风云系列卫星的技术参数。

\begin{table}[htbp]
\centering
\caption{风云系列主要卫星技术参数}
\label{tab:fengyun-satellites}
\begin{tabular}{lccc}
\toprule
卫星名称 & 轨道类型 & 发射时间 & 空间分辨率 \\
\midrule
风云一号D & 极轨 & 2002年 & 1.1 km \\
风云二号H & 静止 & 2018年 & 1 km (可见光) \\
风云三号D & 极轨 & 2017年 & 250 m \\
风云四号A & 静止 & 2016年 & 500 m \\
风云四号B & 静止 & 2021年 & 500 m \\
\bottomrule
\end{tabular}
\end{table}

\subsection{降水观测能力}
风云卫星搭载的微波成像仪和降水测量雷达可以有效监测降水分布。根据雷达方程,降水强度$R$与雷达反射率因子$Z$之间存在经验关系:

\begin{equation}
Z = aR^b
\label{eq:z-r-relation}
\end{equation}

其中,$a$和$b$是经验系数,对于层云降水,典型取值为$a=200$,$b=1.6$。

\section{高分系列对地观测卫星}
高分系列卫星是中国高分辨率对地观测系统的重要组成部分,为水资源管理、环境监测等提供高质量遥感数据。

\subsection{高分一号卫星}
高分一号(GF-1)卫星于2013年发射,搭载2米全色和8米多光谱相机,以及16米多光谱宽幅相机,幅宽达到800公里。

如图\ref{fig:gaofen-coverage}所示,高分系列卫星的不同传感器组合可以满足多种应用需求。

\begin{figure}[htbp]
\centering
\includegraphics[width=0.7\textwidth]{logo/logo.png}
\caption{高分系列卫星覆盖能力示意图}
\label{fig:gaofen-coverage}
\end{figure}

\subsection{高分三号SAR卫星}
高分三号(GF-3)是中国首颗C波段多极化合成孔径雷达卫星,于2016年发射。其雷达成像分辨率最高可达1米,具有12种成像模式。

SAR成像的基本原理基于多普勒效应,目标的径向速度$v_r$与多普勒频移$f_d$的关系为:

\begin{equation}
f_d = \frac{2v_r}{\lambda}
\label{eq:doppler-shift}
\end{equation}

其中,$\lambda$为雷达波长。对于C波段SAR,波长约为5.6 cm。

通过距离-多普勒算法,可以实现二维高分辨率成像。成像分辨率$\rho$与合成孔径长度$L_{SA}$的关系为:

\begin{equation}
\rho_{azimuth} = \frac{L_{antenna}}{2}
\label{eq:sar-resolution}
\end{equation}

其中,$L_{antenna}$为天线物理长度。

\section{遥感数据处理方法}

\subsection{大气校正}
遥感影像在获取过程中受到大气散射和吸收的影响,需要进行大气校正。基于辐射传输方程,地表反射率$\rho_{surface}$的计算公式为:

\begin{equation}
\rho_{surface} = \frac{\pi (L_{sensor} - L_{path})d^2}{E_{sun}\cos\theta_s T_{down}T_{up}}
\label{eq:atmos-correction}
\end{equation}

其中:
\begin{itemize}
\item $L_{sensor}$:传感器接收的辐亮度
\item $L_{path}$:大气路径辐射
\item $d$:日地距离(天文单位)
\item $E_{sun}$:太阳光谱辐照度
\item $\theta_s$:太阳天顶角
\item $T_{down}$:大气下行透过率
\item $T_{up}$:大气上行透过率
\end{itemize}

\subsection{水体提取精度评估}
遥感水体提取结果需要进行精度评估。表\ref{tab:accuracy-metrics}列出了常用的精度评价指标。

\begin{table}[htbp]
\centering
\caption{遥感分类精度评价指标}
\label{tab:accuracy-metrics}
\begin{tabular}{lcc}
\toprule
评价指标 & 计算公式 & 理想值 \\
\midrule
总体精度(OA) & $(TP+TN)/(TP+TN+FP+FN)$ & 1.0 \\
用户精度(UA) & $TP/(TP+FP)$ & 1.0 \\
生产者精度(PA) & $TP/(TP+FN)$ & 1.0 \\
Kappa系数 & $(P_o-P_e)/(1-P_e)$ & 1.0 \\
F1分数 & $2TP/(2TP+FP+FN)$ & 1.0 \\
\bottomrule
\end{tabular}
\end{table}

其中,TP、TN、FP、FN分别表示真正例、真负例、假正例和假负例的数量。

Kappa系数的详细计算公式为:

\begin{equation}
\kappa = \frac{P_o - P_e}{1 - P_e}
\label{eq:kappa}
\end{equation}

其中,$P_o$为观测一致性概率,$P_e$为期望一致性概率:

\begin{align}
P_o &= \frac{TP + TN}{N} \label{eq:po} \\
P_e &= \frac{(TP+FP)(TP+FN) + (TN+FP)(TN+FN)}{N^2} \label{eq:pe}
\end{align}

$N$为样本总数。

\section{应用案例分析}

\subsection{太湖水体监测}
利用高分一号和风云三号卫星数据,可以实现对太湖水体的动态监测。图\ref{fig:taihu-monitoring}展示了多时相水体变化。

\begin{figure}[htbp]
\centering
\includegraphics[width=0.75\textwidth]{logo/logo.png}
\caption{太湖水体多时相遥感监测}
\label{fig:taihu-monitoring}
\end{figure}

水体面积的变化率可以用以下公式计算:

\begin{equation}
\Delta A(\%) = \frac{A_t - A_0}{A_0} \times 100\%
\label{eq:area-change}
\end{equation}

其中,$A_0$为初始时刻水体面积,$A_t$为$t$时刻水体面积。

\section{本章小结}
中国已建立起完整的卫星遥感体系,风云系列气象卫星和高分系列对地观测卫星在水文监测、灾害预警等方面发挥着重要作用。国产卫星的空间分辨率、时间分辨率和光谱分辨率不断提升,为水资源管理提供了可靠的数据支撑。通过综合应用多源遥感数据和先进的处理算法,可以实现对水体的精确监测和动态分析。
